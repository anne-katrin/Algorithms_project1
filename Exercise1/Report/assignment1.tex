\documentclass{article}

\oddsidemargin=0in
\evensidemargin=0in
\textwidth=6in
\topmargin=0in
\textheight=9in

\parindent=0in
\pagestyle{empty}

\usepackage{amsfonts}
\usepackage{amssymb}


\begin{document}

\section*{Chapter 1:  Explanation of the Algorithm} 
 \noindent Before creating sets, the program reads the input file which specifies a certain amount of items, each item is put into a Tuple, which has a size and a value. 
 The tuple is then saved in an ArrayList. As a next step the algorithm creates a power set, where the possible number of tuple combinations are $2^n$. Through bit masking
 and the logical bitwise "and" operator, it is checked if the corresponding bits are both 1, in which case the current tuple is saved in a temporary array list and with help of the 
 shift left operator $<<$ the element is shifted one step to the left. 

\section*{Chapter 2: Algorithm in Pseudocode}
\noindent The following is the pseudocode for the function "createPowerset()". \newline
%just to show how to write this in latex
for i $\rightarrow 2^n$

\section*{Chapter 3: Correctness Justification}

\section*{Chapter 4: Complexity Analysis}

\section*{Chapter 5: Performance Test}

\end{document}
